%        File: documentation.tex
%     Created: Mo Nov 11 02:00  2013 C
% Last Change: Mo Nov 11 02:00  2013 C
%
\documentclass[a4paper]{article}

\usepackage{doi,hyperref,verbatimbox,verbatim,url}
\usepackage[numbers]{natbib}

\newcommand{\plasma}[0]{\texttt{plasma}}
\newcommand{\discrover}[0]{\texttt{discrover}}
\newcommand{\dinuclshuffle}[0]{\texttt{discrover-shuffle}}

\newcommand{\version}[0]{@GIT_DESCRIPTION@}
\newcommand{\theurl}[0]{\url{https://github.com/maaskola/discrover}}
\newcommand{\TikZ}{Ti\emph{k}Z}

\hypersetup{%
  pdfauthor=Jonas Maaskola,
  pdftitle=Manual for discrover v\version{},
  pdfkeywords={Motif searching} {Discriminative learning} {Motif discovery} {Nucleic acids} {RNA} {DNA} {HMM} {Hidden Markov model},
  colorlinks=true     % false: boxed links; true: colored links
}

\title{Manual for \discrover{} v\version{}}
\author{Jonas Maaskola}
\date{\today}
\begin{document}
\maketitle

\begin{abstract}
  This is the documentation for \discrover{} version \version{}.
  Please refer to the publication~\cite{Maaskola2014} for a full description of the method.

  The package provides programs for discriminative motif discovery of protein binding sites in nucleic acid sequences.
  It is available as free software under the GNU General Public License v3 from \theurl{}.

  This document briefly describes the purpose of this motif finding package, explains important concepts, and details usage of the programs that are part of this package.
\end{abstract}

\tableofcontents

\section{Introduction}

Transcriptional and post-transcriptional regulation rely to a large extent on effective mechanisms that allow nucleic acid binding proteins to recognize specific sets of nucleic acids.
Aside from structural cues, binding of regulators is guided by sequence information (motifs) present in cognate nucleic acids.

Motif discovery is the problem of unraveling motifs recognized by a given nucleic acid binding protein from sequences known to harbor occurrences of the motif.
Classically, motif finding was marked by scarcity of data when only few sequences were available.
The introduction of microarray based technologies like ChIP-chip \cite{Ren2000,Iyer2001} and RIP-Chip \cite{Tenenbaum2000,Keene2006} allowed to assay \textit{in vivo} sequence binding specificity on a genome-wide scale.
More recently sequencing based technologies, such as ChIP-Seq \cite{Robertson2007, Johnson2007} and CLIP-Seq \cite{Licatalosi2008,Sanford2009,Hafner2010} further increased the amounts of data yielded by single experiments and simultaneously improved the spatial resolution, reducing uncertainty about the exact location of \textit{in vivo} binding sites.
SELEX \cite{Ellington1990,Tuerk1990} and related sequencing based technologies \cite{Jolma2010}, and protein-binding microarrays (PBMs) \cite{Bulyk2001,Berger2006a} are targeted assays for the \textit{in vitro} sequence binding specificity of nucleic acid binding proteins.

Due to the central importance of the motif discovery problem in computational biology many algorithms addressing it have been developed over the last two decades \cite{Das2007}.
These algorithms employ  a variety of models for the sequence binding specificity of nucleic acid binding proteins, including discrete word-based models, as well as probabilistic models such as position weight matrices (PWM) \cite{Stormo1998}, and hidden Markov models (HMM) \cite{Rabiner1989}.
Word based approaches tend to be computationally efficient and allow fast global optimization, but may fail for motifs that include weak positions \cite{Das2007}.
PWMs can be motivated from biophysical principles \cite{Berg1987,Stormo2000,Foat2006}.
General inference methods for HMMs offer a unified framework for biological sequence modeling \cite{Durbin1998}.
HMMs flexibly model both binding sites and their surrounding sequence context, may account for dependence of neighboring positions via higher order emissions, and length variability of motifs is idiomatically realized via non-constitutive states in the HMM topology.

Because of historically smaller data sizes, many of the presently available motif finding methods were not designed for data sets as large as those produced by current experiments.
Consequently, many of these abort or run unacceptably slow when applied to large data sets.
Thus, even after more than two decades of computational analysis of biological sequences, there is continued interest in the development of new analysis methods that leverage the full potential of large data sets.

\subsection{Overview}
Here we describe a discriminative learning method based on HMMs, realized and available as free software, to automatically discover binding site sequence motifs of nucleic acid binding proteins from arbitrary contrasts,
such as positive and negative example sequences.
Not all of the positive examples need contain binding site occurrences and not all negative examples need be devoid of them.
The framework is applicable to a broad variety of contrasts, comprising the comparison of strongly bound versus weakly bound targets, or of signal sequences with shuffled sequences.
It is also possible to discover context-dependent motifs, or to analyze data sets of different factors for mutually discriminative features.
When available, information from repeat experiments is leveraged by the method.

Putative motifs are identified by a measure of association that quantifies how strongly motif presence varies across conditions of a contrast.
For the chosen objective function, a discrete optimization phase determines IUPAC regular expressions as seeds from which HMMs are initialized.
Subsequently HMM parameters are improved with gradient optimization, which is demonstrated to robustly and effectively function with realistic data sizes.

Objective functions evaluated on multiple contrast may be combined into meta-objectives that allow for a joint analysis of repeat experiments, or to discover motifs that are differential in one contrast but not, or less so, in another.

\section{Concepts}
\subsection{Objective functions}
Motifs are discovered by optimizing an objective function that tries to measure the amount of evidence for the motif.
There are many possible choices for objective functions for this purpose, and this package lets the user decide which objective function to use.

The classical objective function is likelihood.
The corresponding maximum likelihood (ML) principle dictates that parameters are to be chosen such that the probability for the observed data being generated by the statistical model with these parameters is to be as high as possible.

Most of the objective functions considered here are discriminative.
This is to say that they operate on multiple sets of sequences and try to identify motifs that are ``discriminative'' between these sets.
Here, the objective function of course defines exactly what is meant by the term ``discriminative''.

In particular the following objective functions are available:
\begin{itemize}
  \item Likelihood (ML)
  \item Classification posterior probability, also known as maximum mutual information estimation (MMIE)
  \item Difference of occurrence frequency (DFREQ)
  \item Likelihood difference (DLOGL)
  \item Matthews' correlation coefficient (MCC)
  \item Mutual information of condition and motif occurrence (MICO)
\end{itemize}
The publication describing this method, \cite{Maaskola2014}, presents experiments based on synthetic data that suggest that the objective functions MCC, MICO, and MMIE have higher motif discovery performance than the other objective functions analyzed.
\subsection{Seeding}
\subsection{Binding site HMM optimization}

\section{Tutorial}
This tutorial demonstrates the basic usage of the motif discovery tools in this package.
Let us assume that you have one FASTA file with signal sequences, \verb|signal.fa|, and one with control sequences, \verb|control.fa|, and that you want to find motifs of length 8~nt that are discriminative between the two.
\subsection{Basic usage}
To perform a full analysis of discriminative motifs, \discrover{} automatically discovers the most discriminative IUPAC regular expression as seed, initializes an HMM on it, and subsequently optimizes the HMM parameters.
\subsubsection{Analysis of RNA-binding proteins}
To analyze RNA-binding proteins the following command already suffices:\\
\begin{verbbox}
discrover signal.fa control.fa -m 8
\end{verbbox}
\fbox{\theverbbox[t]}

\subsubsection{Analysis of DNA-binding proteins}
In case you want to analyze DNA-binding proteins, please add the \verb|-r| option or its long variant \verb|--revcomp| to also consider motif occurrences on the reverse complementary strand.\\
\begin{verbbox}
discrover signal.fa control.fa -m 8 -r
\end{verbbox}
\fbox{\theverbbox[t]}

\subsection{Manual seed discovery}
\label{tutorial:plasma}
Apart from the fully integrated analysis described above, it is also possible to manually run the seed discovery:\\
\begin{verbbox}
plasma signal.fa control.fa -m 8 -r
\end{verbbox}
\fbox{\theverbbox[t]}

Here, we are looking again for a DNA-binding protein's motif (due to the usage of the option \verb|-r|).
The command line interface of the two programs are intentionally designed to be similar, so that the most important command line options apply to both programs.

\subsubsection{Discovering seeds across a length range}
You can use the following motif specification to discover seeds from a given range of lengths.
Say you want to consider seeds of 5--10~nt length, then you can use this command:\\
\begin{verbbox}
plasma signal.fa control.fa -m 5-10 -r
\end{verbbox}
\fbox{\theverbbox[t]}

Note that the same syntax can also be used for automatic seed discovery with \discrover{}:
\\
\begin{verbbox}
discrover signal.fa control.fa -m 5-10 -r
\end{verbbox}
\fbox{\theverbbox[t]}



\subsection{Manual seed specification for \discrover{}}
Say that you have found a seed of interest for your protein, for example by manually using \plasma{} as described in section~\ref{tutorial:plasma}.
Let us assume it is the IUPAC regular expression \verb|tgtanatw|, where the \verb|n| stands for any nucleotide and the \verb|w| for either \verb|a| or \verb|t|.
You can now use this IUPAC regular expression as seed for an HMM and optimize the HMM parameters as follows:\\
\begin{verbbox}
discrover signal.fa control.fa -m tgtanatw
\end{verbbox}
\fbox{\theverbbox[t]}


\subsection{Finding multiple motifs}
For some applications it is interesting to discover multiple motifs in a datasets.
This allows to reveal motifs of co-factors of the assayed factor.
This is for example of relevance in analyzing ChIP-Seq data, where a given IP'ed transcription factor often acts in conjunction with other factors, whose binding sites are often found nearby, and thus are also frequently enriched in ChIP-Seq sequences.

Discrover has two ways of dealing with multiple motifs, depending on whether or not the switch \verb|--multiple| is used, as described next.
In both modes, multiple seeds are first determined by giving a length range of interest as well as by looking for possibly multiple seeds of each length.
The general syntax for such a motif specification is:\\
\begin{verbbox}
-m 5-16x3
\end{verbbox}
\fbox{\theverbbox[t]}\\
Here, we use the three best seeds for each length of 5--16~nt.

\subsubsection{Multiple independent models with individual seeds}
When the switch \verb|--multiple| is not used, as in the next example, multiple seeds are independently sought, an HMM is initialized and independently optimized for each, and subsequently the single best HMM is reported.\\
\begin{verbbox}
discrover signal.fa control.fa -m 5-16x3 -r
\end{verbbox}
\fbox{\theverbbox[t]}

\subsubsection{HMMs of multiple motifs}
However when the switch \verb|--multiple| is used, as in the next example, a more refined procedure is used.
For the details of this mode please refer to the publication.
In particular, the supplementary material contains flowcharts illustrating how this mode builds models of multiple motifs.\\
\begin{verbbox}
discrover signal.fa control.fa -m 5-16x3 -r --multiple
\end{verbbox}
\fbox{\theverbbox[t]}

Different models are learned and tested when the switch \verb|--multiple| is used.
The final, accepted model's parameters and output files extend the label used for the file names with a \verb|.accepted| tag.

\subsection{Generating sequence logos for motifs}
The package provides multiple ways to generate sequence logos of motifs.
By default, \verb|discrover| will generate sequence logos in PNG and PDF format.
While \verb|plasma| does not by default generate sequence logos, the command line switches \verb|--png| and \verb|--pdf| respectively generate sequence logos in PNG and PDF format.

The same routines can also be used via a separate program: \verb|discrover-logo|.
It creates sequence logos from the \verb|.hmm| files generated by \verb|discrover|, from IUPAC regular expressions, or from motif matrices in files:\\
\begin{verbbox}
discrover-logo motif.hmm
\end{verbbox}
\fbox{\theverbbox[t]}\\
\begin{verbbox}
discrover-logo -i uguahata
\end{verbbox}
\fbox{\theverbbox[t]}\\
\begin{verbbox}
discrover-logo -m matrix.txt
\end{verbbox}
\fbox{\theverbbox[t]}

There is a Ruby script \verb|tikzlogo| that generates a \LaTeX{} document which uses \TikZ{} code to draw a sequence motif, and subsequently compiles the generated \LaTeX{} file into a PDF document.
Finally, the PDF document is converted into a PNG image with the help of \verb|convert| from the ImageMagick package.\\
\begin{verbbox}
tikzlogo motif.hmm
\end{verbbox}
\fbox{\theverbbox[t]}

Note that \verb|tikzlogo| has now been deprecated as it requires a relatively large \LaTeX{} environment to be installed.

\subsection{Using random shuffles as control}
In case you only have a set of signal sequences and you want to use dinucleotide shuffles as control, you have multiple possibilities.
\subsubsection{Generating a set of dinucleotide shuffled sequences}
This software package comes with a C++ implementation of the Altschul-Erkison dinucleotide shuffling algorithm originally implemented in the Python program \verb|altschulEriksonDinuclShuffle.py| by P. Clote, Oct 2003.
It is available in the program \dinuclshuffle{}.
To generate for each sequence contained within a set of signal sequence a shuffled sequence of equal length and equal dinucleotide counts, use the following command:\\
\begin{verbbox}
discrover-shuffle signal.fa > control.fa
\end{verbbox}
\fbox{\theverbbox[t]}

Note that for purposes of reproducibility, you can provide a seed (\verb|--salt|) for the random shuffle generation.

\subsubsection{Automatic generation of random dinucleotide shuffles}
You can also use the same shuffling routines by providing only a set of signal sequences to either \plasma{} or \discrover{}, as the following two commands exemplify.
For \plasma{}\\
\begin{verbbox}
plasma signal.fa -m 8
\end{verbbox}
\fbox{\theverbbox[t]}\\
and for \discrover{}\\
\begin{verbbox}
discrover signal.fa -m 8
\end{verbbox}
\fbox{\theverbbox[t]}

In both cases, dinucleotide shuffles will be generated using the same routines as in \dinuclshuffle{}, and subsequently motif discovery will be executed for motifs that are discriminative between \verb|signal.fa| and the randomly generated set of dinucleotide shuffles.

Note that also in this case for the purpose of reproducibility, you can provide a seed (\verb|--salt|) for the random shuffle generation.



\section{Usage}
There are two main programs that are part of this package, a seed discovery program, \plasma{}, and one for optimizing binding site HMMs, \discrover{}.
The seed discovery program \plasma{} can be used independently, but it is also used by \discrover{} for automatic seed discovery.
The command line options that drive both programs are intentionally similar.

Below, in section~\ref{subsection:running-plasma}, we will first describe how to run \plasma{} independently, before section~\ref{subsection:running-discrover} describes how to use \discrover{}.

\subsection{Seed discovery with \plasma{}}
\label{subsection:running-plasma}
\subsubsection{Getting help on command line options}
An overview of the available command line options is available with the switch \verb|-h| or its long variant \verb|--help|.
See section~\ref{appendix:plasma-cli-help} for a listing of the output of this command with \plasma{} in version \version{}.
\subsubsection{Determining the version}
The command line switch \verb|--version| displays the version of \plasma{}.
If additionally the switch \verb|-v| or its long variant \verb|--verbose| is given, then additionally the full SHA1-string of the version is displayed.

\subsubsection{Seed discovery algorithms}
There are multiple algorithms implemented in \plasma{} that allow to find IUPAC regular expression type seeds.
You can select between them with the switch \verb|--algo|.
It takes one argument, which specifies which algorithm to use.
The following algorithms are provided:
\begin{description}
  \item[Plasma]
    The default and fast algorithm.
    It can be used with the command line switch \verb|--algo plasma|.
    It uses progressive algorithm that maintains a pool of candidates.
  \item[DREME]
    DREME \cite{Bailey2011}, which is part of the MEME motif discovery package, can be used for motif finding if it is installed on the system.
    It can be used by specifying \verb|--algo dreme|.
  \item[Monte-Carlo Markov Chain optimization]
    \plasma{} finds seeds with Monte-Carlo Markov Chain (MCMC) sampling with the command line switch \verb|--algo mcmc|.
    It uses parallel tempering~\cite{Earl2005}, which is also known as replica-exchange, to increase the efficiency of the sampling.
    By default, six parallel chains are used.
\end{description}

\subsection{Binding site HMM optimization with \discrover{}}
\label{subsection:running-discrover}
\subsubsection{Getting help on command line options}
An overview of the available command line options is available with the switch \verb|-h| or its long variant \verb|--help|.
See section~\ref{appendix:discrover-cli-help} for a listing of the output of this command with \discrover{} in version \version{}.

Note that by default not all options are shown.
To display some infrequently used options please specify \verb|-hv| or \verb|--help --verbose|.

In addition, some advanced options are only shown when the option \verb|-V| or \verb|--noisy| is used.
Thus, to display all options please specify \verb|-hV| or their long variants \verb|--help --noisy|.

\subsubsection{Determining the version}
The command line switch \verb|--version| displays the version of \discrover{}.
If additionally the switch \verb|-v| or its long variant \verb|--verbose| is given, then additionally the full SHA1-string of the version is displayed.


\section{Appendix}
\subsection{Command line help of \plasma{}}
\label{appendix:plasma-cli-help}
{%
  \footnotesize
  \verbatiminput{plasma-cli-help.txt}
}

\subsection{Command line help of \discrover{}}
\label{appendix:discrover-cli-help}
{%
  \footnotesize
  \verbatiminput{discrover-cli-help.txt}
}



\nocite{*}
\phantomsection
\addcontentsline{toc}{section}{References}
\bibliographystyle{plainnat}
\bibliography{documentation}


\end{document}
