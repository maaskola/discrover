%        File: documentation.tex
%     Created: Mo Nov 11 02:00  2013 C
% Last Change: Mo Nov 11 02:00  2013 C
%
\documentclass[a4paper]{article}

\usepackage{hyperref,cleveref}

\newcommand{\plasma}[0]{\texttt{plasma}}
\newcommand{\dlhmm}[0]{\texttt{dlhmm}}
\newcommand{\version}[0]{@GIT_DESCRIPTION@}

\title{Documentation for \dlhmm{} v\version{}}
\author{Jonas Maaskola}
\date{\today}
\begin{document}
\maketitle

\begin{abstract}
  This is the documentation for \dlhmm{} version \version{}.
  Please refer to the publication~\cite{Maaskola2013} for a full description of the method.

  This document briefly describes the purpose of this motif finding package, explains important concepts, and details usage of the programs that are part of this package.
\end{abstract}

\tableofcontents

\section{Introduction}

\section{Concepts}
\subsection{Objective functions}
\subsection{Seeding}
\subsection{Binding site HMM optimization}

\section{Usage}
There are two main programs that are part of this package, a seed discovery program, \plasma{}, and one for optimizing binding site HMMs, \dlhmm{}.
The seed discovery program \plasma{} can be used independently, but it is also used by \dlhmm{} for automatic seed discovery.
The command line options that drive both programs are intentionally similar.

Below, in \cref{subsection:running-plasma}, we will first describe how to run \plasma{} independently, before \cref{subsection:running-dlhmm} describes how to use \dlhmm{}.

\subsection{Seed discovery}
\label{subsection:running-plasma}
\subsubsection{Command line help}
An overview of the available command line options is available with the switch \lstinline|-h| or its long variant \lstinline|--help|.
See \cref{appendix:plasma-cli-help} for a listing of the output of this command with \plasma{} in version \version{}.


\subsection{Binding site HMM optimization}
\label{subsection:running-dlhmm}
\subsubsection{Command line help}
An overview of the available command line options is available with the switch \lstinline|-h| or its long variant \lstinline|--help|.
See \cref{appendix:dlhmm-cli-help} for a listing of the output of this command with \dlhmm{} in version \version{}.

\section{Appendix}
\subsection{Command line help of \plasma{}}
\label{appendix:plasma-cli-help}
\begin{verbatim}
@PLASMA_CLI_HELP_OUTPUT@
\end{verbatim}

\subsection{Command line help of \dlhmm{}}
\label{appendix:dlhmm-cli-help}
\begin{verbatim}
@DLHMM_CLI_HELP_OUTPUT@
\end{verbatim}



\nocite{*}
\addcontentsline{toc}{section}{References}
\bibliographystyle{plain}
\bibliography{documentation}
* corresponding author



\end{document}


